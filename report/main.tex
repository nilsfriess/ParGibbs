\documentclass[
fontsize=11pt,
paper=a4,
numbers=noenddot,
parskip=half
]{scrartcl}

\usepackage[bottom=3cm, top=3cm]{geometry}

\usepackage[utf8]{inputenc}
\usepackage[T1]{fontenc}
\usepackage{microtype}

\usepackage[backend=biber]{biblatex}
\addbibresource{bibliography.bib}

\usepackage{hyperref}  

\usepackage{mathtools}
\usepackage{amsthm}
\usepackage{amssymb}
\usepackage{bm}

\usepackage{todonotes}
\usepackage[ruled,vlined,resetcount]{algorithm2e}
\SetKwInput{KwData}{Input}


\addtokomafont{disposition}{\rmfamily}

\title{Parallel Multigrid Monte Carlo\\
{\normalsize Software practical report}}
\date{\today}
\author{Nils Friess}

\begin{document} 
\maketitle

\begin{abstract}
    This report introduces a parallel implementation of the Multigrid Monte Carlo (MGMC) method for sampling from Gaussian distributions with large and sparse precision matrix on distributed memory machines. MGMC is in a certain sense the stochastic counterpart of the classic geometric Multigrid method, and a parallel implementation of MGMC can reuse large parts of a parallel implementation of the classic Multigrid method with the exception of the smoothers which are replaced by certain random samplers. Our implementation reuses Multigrid components from the \emph{Portable and Extensible Toolkit for Scientific computing (PETSc)} and implements the following parallel random samplers within this framework: a multicolour Gibbs sampler which is particularly suited for distributions corresponding to Gaussian Markov random fields on simple grids or lattices; a border/interior Gibbs sampler which is advantageous if the underlying grid is unstructured where a multicolouring approach becomes inefficient; a so called \emph{Hogwild} Gibbs sampler which has the highest parallel efficency but does not target the desired distribution.  
\end{abstract}

\printbibliography

\end{document}

 
%%% Local Variables:
%%% mode: latex
%%% TeX-master: t
%%% TeX-command-extra-options: "-shell-escape"
%%% End:
